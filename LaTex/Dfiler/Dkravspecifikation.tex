\chapter{Kravspecifikation}

\begin{longtabu} to \linewidth{@{}l l l X[j]@{}}
    Version &    Dato &    Ansvarlig &    Beskrivelse\\[-1ex]
    \midrule

\label{version_Systemark}
\end{longtabu}


\section{Indledning}
Kravspecifikationen vil beskrive, ud fra en række modeller, hvordan EKG-systemet fungerer. Helt generelt er EKG-måling en simpel metode, til at måle hjertets elektriske aktivitet via elektroder, som registrerer elektriske impulser, placeret på huden. Ud fra disse impulser dannes en graf, som benyttes til at analysere hjertets funktionalitet ud fra P-, Q-, R-, S- og T-takkerne, og dermed konkludere om den pågældende patient har et raskt eller sygt hjerte, samt hvilken sygdom, der er tale om. Helt specifikt for denne opgave er formålet, at identificere sygdommen atrieflimmer via et virtuelt EKG-signal. 

\section{Funktionelle krav}
De funktionelle krav vil nedenstående beskrives ud fra Aktør-kontekstdiagram, aktørbeskrivelse, Use Cases samt Use Case diagram. 

\subsection{Aktør-kontekstdiagram}



\subsection{Aktørbeskrivelse}

\begin{table}[H]
\begin{tabularx}{\textwidth}{l l X}
     Aktørnavn  & Type      & Beskrivelse \\ \midrule
                                                                                                                                                                        
   
     \bottomrule                                                                                                                   
    \end{tabularx}
    \caption {Aktørbeskrivelse}
    \label{tab:aktoerbeskrivelse}
	
\end{table}

\subsection{Use case-diagram}


\subsection{Use Cases}

\begin{longtabu} to \linewidth{@{}l r X[j]@{}} %UC1%
    {\large \textbf{Use Case 1}} && \\
    \toprule
    Navn &&    Log ind\\
    Use case ID &&    1\\
    Samtidige forløb &&    1\\
    Primær aktør &&    Brugeren\\
    Initialisere &&    Brugeren ønsker at logge ind\\
    Forudsætninger &&  At der er logget ud efter en tidligere måling\\
    Resultat &&    Brugeren bliver logget på og kan foretage en måling                     \\ \midrule
    Hovedforløb &    1. &    Brugeren indtaster username samt password\\[-1ex]   						 	
                &    2. &    Brugeren trykker på "Login"\--knappen. Login-vinduet lukkes ned mens CPR-vinduet åbnes\newline
                	[2.a \textit{Username eller password er forkert}]\\[-1ex]
                &    \\ \midrule
                
    Undtagelser &    2a. & Besked vises på skærmen med tekst, der informerer om, at username eller password er forkert. Der forsættes i UC1 ved punkt 1     \\ \bottomrule
\caption{Fully dressed Use Case 1.}
\label{UC1}
\end{longtabu}


\section{Ikke-funktionelle krav}


\subsection{(F)URPS+}
MoSCoW er angivet i parentes med hhv. M, S, C eller W.

\textbf{Usability}
\begin{itemize}
	\item (M) Brugeren skal kunne starte en default-måling maksimalt 20 sek. efter opstart af programmet
	\item (M) Login-vinduet skal indholde en "login"\--knap til at logge på og få vist EKG-vinduet
	\item (M) EKG-vinduet skal indeholde en "start"\--knap til at igangsætte målingerne
	\item (M) EKG-vinduet skal indeholde en "log ud"\--knap
	\item (M) EKG-vinduet  skal indeholde en "gem"\--knap
	\item (M) Information-vinduet skal indeholde en "gem"\--knap
\end{itemize}

\textbf{Reliability}
\begin{itemize}
	\item (M) Systemet skal have en effektiv MTBF (Mean Time Between Failure) på 20 minutter og en MTTR (Mean Time To Restore) på 1 minut.
				\begin{align}
					Availability = \frac{MTBF}{MTBF+MTTR} = \frac{20}{20+1} = 0,952 = 95,2 \%
				\end{align}

\end{itemize}

\textbf{Performance}
\begin{itemize}
	\item (M) Der skal vises en EKG-graf i EKG-vinduet, hvor spænding vises op af y-aksen (-1V til 1V) og tiden på x-aksen
	\item (M) Grafen skal være scrollbar på x-aksen, så brugeren selv ved brug af musen kan vælge det udsnit af grafen, der skal vises mere detaljeret
	\item (M) Skal tage en sample over et brugerbestemt interval, hvor frekvensen  er tilpasset målingerne, således at grafen er analyserbar
\end{itemize}

\textbf{Supportability}
\begin{itemize}
	\item (M) Softwaren er opbygget af trelagsmodellen
\end{itemize}















