\chapter{Kravspecifikation}

\begin{longtabu} to \linewidth{@{}l l l X[j]@{}}
    Version &    Dato &    Ansvarlig &    Beskrivelse\\[-1ex]
    \midrule

\label{version_Systemark}
\end{longtabu}


\section{Indledning}


\section{Funktionelle krav}
De funktionelle krav vil nedenstående beskrives ud fra Aktør-kontekstdiagram, aktørbeskrivelse, Use Cases samt Use Case diagram. 

\subsection{Aktør-kontekstdiagram}



\subsection{Aktørbeskrivelse}

\begin{table}[H]
\begin{tabularx}{\textwidth}{l l X}
     Aktørnavn  & Type      & Beskrivelse \\ \midrule
                                                                                                                                                                        
   
     \bottomrule                                                                                                                   
    \end{tabularx}
    \caption {Aktørbeskrivelse}
    \label{tab:aktoerbeskrivelse}
	
\end{table}

\subsection{Use case-diagram}


\subsection{Use Cases}

\begin{longtabu} to \linewidth{@{}l r X[j]@{}} %UC1%
    {\large \textbf{Use Case 1}} && \\
    \toprule
    Navn &&    Start blodtryksmåling med digitalt filter\\
    Use case ID &&    1\\
    Samtidige forløb &&    1\\
    Primær aktør &&    Bruger\\
    Sekundære aktør &&	Borger \\
    Mål &&    Bruger ønsker at starte en blodtryksmåling med et digitalt filter\\
    Initiering &&	Startes af Bruger\\
    Forudsætninger &&  Borger er tilsluttet systemet. Systemet er aktivt og tilgængeligt. Det digitale filter er slået fra (Kalibrering)  \\
    Resultat &&		Det filtrerede blodtrykssignal bliver vist kontinuerligt på brugergrænsefladen                         \\ \midrule
    Hovedforløb &    1. &    Brugeren slår digitalt filter til på switch\\[-1ex]   						 	
                &    2. &    Brugeren trykker på "Start"\\[-1ex]
                &    3.	&	 Blodtrykssignalet vises på brugergrænsefladen\newline\\ \midrule
                
    Undtagelser &     &      \\ \bottomrule
\caption{Fully dressed Use Case 1.}
\label{UC1}
\end{longtabu}


\begin{longtabu} to \linewidth{@{}l r X[j]@{}} %UC2%
    {\large \textbf{Use Case 2}} && \\
    \toprule
    Navn &&    Start blodtryksmåling uden digitalt filter\\
    Use case ID &&    2\\
    Samtidige forløb &&    1\\
    Primær aktør &&    Bruger\\
    Sekundære aktør &&	Borger \\
    Mål &&    Bruger ønsker at starte en blodtryksmåling uden et digitalt filter\\
    Initiering &&	Startes af Bruger\\
    Forudsætninger &&  Borger er tilsluttet systemet. Systemet er aktivt og tilgængeligt. Det digitale filter er slået til (Kalibrering)  \\
    Resultat &&		Det ufiltrerede blodtrykssignal bliver vist kontinuerligt på brugergrænsefladen                         \\ \midrule
    Hovedforløb &    1. &    Brugeren slår digitalt filter fra på switch\\[-1ex]   						 	
                &    2. &    Brugeren trykker på "Start"\\[-1ex]
                &    3.	&	 Blodtrykssignalet vises på brugergrænsefladen\newline\\ \midrule
                
    Undtagelser &     &      \\ \bottomrule
\caption{Fully dressed Use Case 2.}
\label{UC2}
\end{longtabu}


\begin{longtabu} to \linewidth{@{}l r X[j]@{}} %UC3%
    {\large \textbf{Use Case 3}} && \\
    \toprule
    Navn &&    Optag lydsekvens relateret til blodtryksmålingen\\
    Use case ID &&    3\\
    Samtidige forløb &&   *\\
    Primær aktør &&    Bruger\\
    Sekundære aktør &&	Borger \\
    Mål &&    Bruger ønsker at optage lydsekvens til blodtryksmålingen\\
    Initiering &&	Startes af Bruger\\
    Forudsætninger &&  Systemet er aktivt og tilgængeligt. UC1 eller UC2 er igang  \\
    Resultat &&		Lydsekvens relateret til blodtryksmåling er optaget                 \\ \midrule
    Hovedforløb &    1. &    Brugeren trykker på "Start/stop optag"\--knappen for at starte lydoptagelsen\\[-1ex]   						 	
                &    2. &    Systemet optager lydsekvens og "Optag"\--diode lyser\\[-1ex]
                &    3.	&	 Bruger trykker på "Start/stop optag"\--knappen for at stoppe optagelsen\newline\\ \midrule
                
    Undtagelser &     &      \\ \bottomrule
\caption{Fully dressed Use Case 3.}
\label{UC3}
\end{longtabu}


\begin{longtabu} to \linewidth{@{}l r X[j]@{}} %UC4%
    {\large \textbf{Use Case 4}} && \\
    \toprule
    Navn &&    Gem blodtryksmåling og lydsekvens\\
    Use case ID &&    4\\
    Samtidige forløb &&    1\\
    Primær aktør &&    Bruger\\
    Sekundære aktør &&	Borger \\
    Mål &&    Bruger ønsker at gemme en blodtryksmåling og lydsekvens\\
    Initiering &&	Startes af Bruger\\
    Forudsætninger &&  Systemet er aktivt og tilgængeligt. UC1 eller UC2 er gennemført  \\
    Resultat &&		Blodtryksmåling og lydsekvens bliver gemt i Database                   \\ \midrule
    Hovedforløb &    1. &    Brugeren trykker på "Gem"\--knappen\\[-1ex]   						 	
                &    2. &    Systemet gemmer blodtryksmåling og lydsekvens og 		 "Gem"\--diode lyser \newline
                			 [2.a \textit{Bruger har ikke optaget lydsekvens}]\\ 
                			 \midrule
                
    Undtagelser &   [Extension 1: ]  &      \\ \bottomrule
\caption{Fully dressed Use Case 4.}
\label{UC4}
\end{longtabu}



\section{Ikke-funktionelle krav}


\subsection{(F)URPS+}
MoSCoW er angivet i parentes med hhv. M, S, C eller W.

\textbf{Usability}
\begin{itemize}
	\item (M) Brugeren skal kunne starte en default-måling maksimalt 20 sek. efter opstart af programmet
	\item (M) Login-vinduet skal indeholde en "login"\--knap til at logge på og få vist EKG-vinduet
	\item (M) EKG-vinduet skal indeholde en "start"\--knap til at igangsætte målingerne
	\item (M) EKG-vinduet skal indeholde en "log ud"\--knap
	\item (M) EKG-vinduet  skal indeholde en "gem"\--knap
	\item (M) Information-vinduet skal indeholde en "gem"\--knap
\end{itemize}

\textbf{Reliability}
\begin{itemize}
	\item (M) Systemet skal have en effektiv MTBF (Mean Time Between Failure) på 20 minutter og en MTTR (Mean Time To Restore) på 1 minut.
				\begin{align}
					Availability = \frac{MTBF}{MTBF+MTTR} = \frac{20}{20+1} = 0,952 = 95,2 \%
				\end{align}

\end{itemize}

\textbf{Performance}
\begin{itemize}
	\item (M) Der skal vises en EKG-graf i EKG-vinduet, hvor spænding vises op af y-aksen (-1V til 1V) og tiden på x-aksen
	\item (M) Grafen skal være scrollbar på x-aksen, så brugeren selv ved brug af musen kan vælge det udsnit af grafen, der skal vises mere detaljeret
	\item (M) Skal tage en sample over et brugerbestemt interval, hvor frekvensen  er tilpasset målingerne, således at grafen er analyserbar
\end{itemize}

\textbf{Supportability}
\begin{itemize}
	\item (M) Softwaren er opbygget af trelagsmodellen
\end{itemize}















