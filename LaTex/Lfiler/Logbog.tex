\chapter{Logbog}

\section{Dato: 08-09-2015 }
\hrule

\textbf{Omhandler:} Udarbejdelse af samarbejdskonkrakt 

\textbf{Ansvarlig:} Lise, Nina, Toke, Anders og Jakob

\textbf{Logbog}
\\
\\
Dagsorden:
\begin{itemize}
	\item Udarbejdelse af samarbejdsaftale
	\item Forklaring af Github
	\item Planlægning af vejledermøde
\end{itemize}

Samarbejdsaftalen er udarbejdet og er blevet lagt på Github til godkendelse af alle gruppens medlemmer.\newline 
Lise introducerede Github til gruppens medlemer.\newline 
Annsofie skriver til vores vejleder angående møde mandag d. 14-09-2015 kl. 10.15\newline 
Til næste møde skal "vejledning til 3. semester projekt" være læst og forstået.\newline





\section{Dato: 21-09-2015 }
\hrule

\textbf{Omhandler:} Kravspecifikation

\textbf{Ansvarlig:} Lise, Nina, Annsofie, Toke, Anders og Jakob

\textbf{Logbog}
\\
\\
Dagsorden:
\begin{itemize}
	\item Kravspecifikation

\end{itemize}

Vi vil implementere systole og diastole på vores blodtryksmåler.
En knap i GUI'en til at bestemme om filteret skal være aktivt eller passivt.
En knap i GUI'en til at kalibrere.
En 'Start/record'-knap der også optager lyd de første 30 sek. eks.
EN 'Start'-knap til at starte blodtryksmåling.
Vi dropper login funktionen.
Vi talte om en form for talefunktion, der optager når lægen råber patientens informationer op. Den lydfil skal så kobles til den blodtryksmåling, som foretages fx. i ambulancen. 
Vi skal have defineret: System, borger, UI


\section{Dato: xx-09-2015 }
\hrule

\textbf{Omhandler:}  

\textbf{Ansvarlig:} 

\textbf{Logbog}
\\
\\
Dagsorden:
\begin{itemize}
	\item 
	\item 
	\item 
\end{itemize}