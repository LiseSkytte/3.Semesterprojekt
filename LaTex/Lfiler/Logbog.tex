\chapter{Logbog}

% Logbog for den 08-09-2015
\section{Dato: 08-09-2015 }
\hrule

\textbf{Omhandler:} Udarbejdelse af samarbejdskonkrakt 

\textbf{Ansvarlig:} Lise, Nina, Toke, Anders og Jakob

\textbf{Logbog}
\\
\\
\textbf{Dagsorden:}
\begin{itemize}
	\item Udarbejdelse af samarbejdsaftale
	\item Forklaring af Github
	\item Planlægning af vejledermøde
\end{itemize}

Samarbejdsaftalen er udarbejdet og er blevet lagt på Github til godkendelse af alle gruppens medlemmer.\\
Lise introducerede Github til gruppens medlemer.\\
Annsofie skriver til vores vejleder angående møde mandag d. 14-09-2015 kl. 10.15\\
Til næste møde skal "vejledning til 3. semester projekt" være læst og forstået.\\
	



% Logbog for den 21-09-2015
\section{Dato: 21-09-2015 }
\hrule

\textbf{Omhandler:} Kravspecifikation

\textbf{Ansvarlig:} Lise, Nina, Annsofie, Toke, Anders og Jakob

\textbf{Logbog}
\\
\\
\textbf{Dagsorden:}
\begin{itemize}
	\item Kravspecifikation
\end{itemize}

Vi vil implementere systole og diastole på vores blodtryksmåler.\\
En knap i GUI'en til at bestemme om filteret skal være aktivt eller passivt.\\
En knap i GUI'en til at kalibrere.\\
En 'Start/record'-knap der også optager lyd de første 30 sek. eks.\\
EN 'Start'-knap til at starte blodtryksmåling.\\
Vi dropper login funktionen.\\
Vi talte om en form for talefunktion, der optager når lægen råber patientens informationer op. Den lydfil skal så kobles til den blodtryksmåling, som foretages fx. i ambulancen. \\
Vi skal have defineret: System, borger, UI




% Logbog for den 23-09-2015
\section{Dato: 23-09-2015 }
\hrule

\textbf{Omhandler:} Kravspecfikation 

\textbf{Ansvarlig:} Lise, Nina, Annsofie, Toke, Anders og Jakob

\textbf{Logbog}
\\
\\
\textbf{Dagsorden:}
\begin{itemize}
	\item Definationer
	\item Aktør-kontekst diagram
	\item Use Case diagram
	\item Use Case forløb
\end{itemize}

Ordliste:\\ 
Use Case 1 = UC1 \\
PhysioNet + Analog-Discovery = Borger, som er en Package \\
Analog-Discovery = AD \\

Der er blevet lavet et udkast til AK og UC-diagram. \\
Vi har en Bruger som primær aktør. Vi har lavet en package med PhysioNet og AD, som repræsentere Borger. DAQ er en sekundær aktør, der påvirker Borger. Databasen er også en sekundær aktør. 

Vision for BT-måler:\\
Apparatet skal være fungerende som en "normal" BT-måler med mulighed for visning af kontinuerligt BT med/uden filter. Derudover skal der være mulighed for at optage lydsekvenser direkte på apperatet som kan bruges som metadata til målingen, uden at bruger skal indtaste data på UI. Dette skal implementeres i en Rec. knap der tænder for mikrofonen og optagen en lydsekvens på f.eks. 30 sek, og derefter slukker. Denne lydsekvens kunne bruges til at dokumentere information om Borger. Disse kunne være alder, køn, traume, ABCDE (førsthjælp), mv. på Borger der monitoreres. \\
Når blodtryksmålingen gemmes på database eller fil, bliver lydsekvensen gemt sammen med denne. Dette kan anvendes til forskning senere. 

Vi ønsker også at vis dia/sys i UI. 


	

% Logbog for den 29-09-2015
\section{Dato: 29-09-2015 }
\hrule

\textbf{Omhandler:} Kravspecifikation 

\textbf{Ansvarlig:} Alle 

\textbf{Logbog}
\\
\\
\textbf{Dagsorden:}
\begin{itemize}
\item Lave UC's om samt AK- og UC-diagammer
\end{itemize}

Efter mødet med Peter i torsdags måtte vi indse, at ideen med optagelse ikke var nogen god ide - eller Peter mente i hvert fald ikke, det var nødvendigt. Han vil meget heller have, at vi kigger forskningsbaseret på opgaven og holder os indenfor de krav, der er stillet til projektet.\\
Derfor er vi gået over til, at holde det til kravene. Derfor har vi udarbejdet nogle nye UC's - der er kommet 5 indtil videre. 
\begin{itemize}
	\item Vis måling 
	\item Kalibrer
	\item Nulpunktjustér 
	\item Aktivér digitalt filter
	\item Gem 
\end{itemize}

Disse fik vi udformet og aftalte at mødes i morgen igen, for at få alle de ikke-funktionelle krav og AT. 
\\
Anders og Lise gik lidt igang med at kigge på hardwarddelen. Altså har siddet også udregnet lidt på, hvad det er for nogle komponentværdier, vi skal arbejde med for at designe vores lavpasfiltre.  


	
	
% Logbog for den 30-09-2015
\section{Dato: 30-09-2015 }
\hrule

\textbf{Omhandler:} Kravspecifikation og Accepttest  

\textbf{Ansvarlig:} Alle 

\textbf{Logbog}
\\
\\
\textbf{Dagsorden:}
\begin{itemize}
\item Ændring af UC's
\end{itemize}
	
Igen snakkede vi lidt om funktionaliteten af vores produkt og kom frem til at der skulle ændres lidt igen. Nu har vi altså 6 UC, hvor vi også byttede om på 1 og 2. Så nu hedder det: 
\begin{itemize}
	\item Kalibrer
	\item Vis måling 
	\item Nulpunktjustér 
	\item Deaktivér filter 
	\item Aktivér filter
	\item Gem 
\end{itemize}

Ikke-funktionelle krav er nedskrevet i forhold til (F)RUPS(+). Accepttest er også blevet lavet færdig til Review. 

	
	

% Logbog for den 01-10-2015
\section{Dato: 01-10-2015 }
\hrule

\textbf{Omhandler:} Brødtekst til Kravspecifikation 

\textbf{Ansvarlig:} Lise 

\textbf{Logbog}
\\
\\
\textbf{Dagsorden:}
\begin{itemize}
	\item Brødtekst til indledning
\end{itemize}

Lavede lige hurtig lidt brødtekst til indledning, AK-diagram samt UC-diagram. 




% Logbog for den 06-10-2015
\section{Dato: 06-10-2015 }
\hrule

\textbf{Omhandler:} Review, tidsplan og hardware  

\textbf{Ansvarlig:} Nina, Lise, Toke, Anders og Jakob

\textbf{Logbog}
\\
\\
\textbf{Dagsorden:}
\begin{itemize}
	\item Review af gruppe 2's KS og AT
	\item Udvikling af tidsplan
	\item Videreudvikling af hardware
\end{itemize}

Der er blevet lavet review af gruppe 2's KS og AT, og givet kommentarer hertil. \\
Der er i gruppen blevet udarbejdet en tidsplan, hvori deadlines fra IHA er inkluderet for accepttest, kravspecifikation, design, accepttest med vejleder aflevering. Dertil har gruppen selv tilføjet en række deadlines for software, hardware, rapportskrivning, dokumentation og korrekturlæsning. Alle deadlines er lagt ind i en tabel for at skabe overblik, således at tidsplanen kan benyttes som pejlmærke for hvad vi skal lave og hvornår.\\
Der er desuden blevet udarbejdet en detaljeret tidsplan, hvor vi beskriver det grundlæggende for udviklingsprocessen.\\
Anders og Toke arbejdede videre med hardware delen på fumlebrættet.
Forsøger at aftale et møde med Peter mandag eller tirsdag efter kl. 12.00 i uge 43.

	
	

% Logbog for den 08-10-2015
\section{Dato: 08-10-2015 }
\hrule

\textbf{Omhandler:} Review af kravspecifikation og accepttest 

\textbf{Ansvarlig:} Alle

\textbf{Logbog}
\\
\\
\textbf{Dagsorden:}
\begin{itemize}
	\item Overskrift til versionshistorik
	\item AK - bruger og borger ligger tæt op af hinanden (brug evt. forsker eller lignende.
	\item Specificere at det kun er prototype, således at 'borger' giver mening.
	\item Mangler lidt til fremtiden - hvem skal den kunne bruges til der?
	\item Der er uklarhed mht det digitale filter. Specielt i UC2.
	\item UC5 rager der også uklarhed.
	\item UC3: mangler en afslutning - det fremgår af GUI at ...
	\item UC6: lidt uklarhed omkring stjernen i samtidig forløb
	\item Sidehoved er blevet genbrugt - HUSK AF ÆNDRE DET!
	\item Undtagelse: UC6 forsættes ved punkt 6***
	\item Lidt forvirret over interfacet. Hvor mange grafer bliver der vist? 
	\item Punkt 4 i Interface: Lidt uklart hvor tallene bliver vist?
	\item Der er stor forvirring omkring 2.2 og specielt under Andet+
	\item Lav vores test af kalibrering lidt mere idiotsikker. 
\end{itemize}

Vi rettede efterfølgende de ting, som vi mente skulle rettes.

	


% Logbog for den 19-10-2015
\section{Dato: 19-10-2015 }
\hrule

\textbf{Omhandler:} Design 

\textbf{Ansvarlig:} Annsofie, Nina, Jakob og Lise

\textbf{Logbog}
\\
\\
\textbf{Dagsorden:}
\begin{itemize}
	\item Udarbejdelse af BDD diagram for systemet
	\item Udarbejdelse af SD'er(generel form) for alle use cases
	\item Videreudvikling af hardwaredelen
\end{itemize}

Der er blevet startet på BDD diagrammet for systemet + beskrivelsen. Ikke færdiggjort.\\
Der blevet lavet udkast for SD i generel form til samtlige use cases. 




% Logbog for den 20-10-2015
\section{Dato: 20-10-2015 }
\hrule

\textbf{Omhandler:} Design 

\textbf{Ansvarlig:} Lise, Jakob, Nina og Annsofie

\textbf{Logbog}
\\
\\
\textbf{Dagsorden:}
\begin{itemize}
	\item BDD og IBD
	\item Sekvens diagram
\end{itemize}

Der er blevet lavet om på BDD, så der nu er ports med. Der er også blevet lavet et IDB. 
\\
Vi har snakket med Kim om vores SD, og vi fandt ud af, at vi havde lavet en blanding af de to forskellige sekvensdiagrammer. Vi har valgt, at vi vil lave sekvensdiagrammer for softwaren (dem der er i applikationsmodellen) dog for hver UC.
\\
I morgen skal vi hele gruppe side og snakke om sekvensdiagrammerne, så vi bliver enige om, hvordan softwaren skal bygges op.  
	
	
	

% Logbog for den 27-10-2015
\section{Dato: 27-10-2015 }
\hrule

\textbf{Omhandler:} Rettelse af use cases, ak-diagram og kravspec

\textbf{Ansvarlig:} Jakob, Toke, Nina, Annsofie, Lise 

\textbf{Logbog}
\\
\\
\textbf{Dagsorden:}
\begin{itemize}
	\item Use cases
	\item AK-diagram
	\item Kravspec
\end{itemize}

Husk at skriv en uddybende tekst til trykstranduceren ift. AK-diagram. Det skal specificeres at den ikke benyttes(indgår i systemet) når der testes med PN og AD.
\\
Der blev opdateret use case diagrammer, aktør kontekst diagram og tilføjet et og ændret et punkt til ikke-funktionelle krav (MTBF og nulpunktsjustering), samt kigget nærmere på udvikling af en overordnet systembeskrivelse vha. BDD og IBD. Derudover har vi også kigget nærmere på vores HW BDD og IBD.
\\
Vi begyndte på domænemodellen og fik lavet et solidt udgangspunkt til klassediagrammet. 
\\
Vi mangler stadig at opdatere nogle de ikke-funktionelle krav ift. de GUI krav vi gerne vil have. Tjek mødereferat fra 26.10.15.




% Logbog for den 28-10-2015
\section{Dato: 28-10-2015 }
\hrule

\textbf{Omhandler:} Anders har fået Texpad til at virke. 

\textbf{Ansvarlig:} Anders
 
\textbf{Logbog}
\\
\\
\textbf{Dagsorden:}
\begin{itemize}
	\item Nu skal der skrives logbøger.
\end{itemize} 

Dag 28. Jeg har omsider fået Texpad til at typesette. Dette medfører at jeg kan skrive logbog. Dette er et lille skidt for én mand, men et stort skridt for semesterprojektgruppen.




% Logbog for den 02-11-2015
\section{Dato: 02-11-2015 }
\hrule

\textbf{Omhandler:} Udarbejdelse af design  

\textbf{Ansvarlig:} Lise, Annsofie, Toke, Anders og Jakob

\textbf{Logbog}
\\
\\
\textbf{Dagsorden:}
\begin{itemize}
	\item Domænemodel
	\item Skrive tekster til SysML diagrammer
\end{itemize}

Vi har udarbejdet en domænemodel, samt skrevet uddybbende tekster til design/diagrammer. \\
Klassediagrammer er blevet skrevet ind i Visio. 




% Logbog for den 04-11-2015
\section{Dato: 04-11-2015 }
\hrule

\textbf{Omhandler:} Udarbejdelse af design  

\textbf{Ansvarlig:} Lise, Annsofie, Nina, Toke, Anders og Jakob

\textbf{Logbog}
\\
\\
\textbf{Dagsorden:}
\begin{itemize}
	\item Domænemodel
	\item Skrive tekster til ISE diagrammer
	\item BDD og IBD til Hardware delen
	\item Rettelse af Aktør-kontekst diagram
	\item Rettelse af Use Cases
	\item Rettelse af diagrammer ift. at trykstranduceren skal være en aktør.
	\item Software
\end{itemize}

Domænemodellen er blevet redigeret og færdiggjort til review.\\
BDD og IBD til hardware er lavet færdig og klar til review.\\
Efter krav om at trykstranduceren skal være en aktør, er diagrammer der indeholder aktøren "Måleobjekt" blevet redigeret, så de stemmer overens.\\
Der er blevet arbejdet med softwaren og lavet et udkast til et "Gem-vindue".\\
Der er lavet en versionshistorik for designfasen.




% Logbog for den 12-11-2015
\section{Dato: 12-11-2015 }
\hrule

\textbf{Omhandler:} Review 

\textbf{Ansvarlig:} Lise, Annsofie, Nina, Toke, Anders og Jakob

\textbf{Logbog}
\\
\\
\textbf{Dagsorden:}
\begin{itemize}
	\item Rettelser af design efter review
\end{itemize}




% Logbog for den 17-11-2015
\section{Dato: 17-11-2015 }
\hrule

\textbf{Omhandler:} Rettelse af dokumentation efter review 

\textbf{Ansvarlig:} Lise, Annsofie, Nina, Toke, Anders og Jakob

\textbf{Logbog}
\\
\\
\textbf{Dagsorden:}
\begin{itemize}
	\item Skabe overblik over hvad der skal rettes
\end{itemize}

Vi har idag fået rettet vores Use Cases og tilhørende accepttests en lille smule.\\
Domænemodellen er blevet redigeret igen efter review med en anden gruppe.\\
Vi har rettet design igennem og færdiggjort HW delen hvad angår både design og implementering og test.\\
Vi har desuden fået skabt et overblik over hvilke "små" ændringer der skal laves gennem hele dokumentationen:
\begin{itemize}
	\item Ordliste: Signalbehandling
	\item BDD: Trykmonitor --> tryktransducer
	\item 2.2 Grænseflader - orden tekst + forkortelser
	\item 2.33 HW - jord --> 0V
	\item Tryktransducer temperatur - 10-50 grader C
	\item Kondensator 340 nF - på lager? 
	\item 2.35 - byt filter- og forstærkerblok
	\item Indgangsimpedans uendelig - kun i teorien
	\item Ny domænemodel - laves ud fra use cases
	\item Nye sekvensdiagrammer
	\item 2.3. Forstærkerblok - ordvalg?
	\item Klassediagrammer skal opdateres
	\item Test af forstærkerblok
\end{itemize} 




% Logbog for den 01-12-2015
\section{Dato: 01-12-2015 }
\hrule

\textbf{Omhandler:} Rapport

\textbf{Ansvarlig:} Lise

\textbf{Logbog}
\\
\\
Rapport dokumentet er blevet oprettet med de forskellige afsnit den skal indeholde. 




% Logbog for den 02-12-2015
\section{Dato: 02-12-2015 }
\hrule

\textbf{Omhandler:} 

\textbf{Ansvarlig:} Nina, Lise, Annsofie, Toke, Anders og Jakob

\textbf{Logbog}
\\
\\
\textbf{Dagsorden:}
\begin{itemize}
	\item Ret use case navne mv.
	\item Lav sq diagrammer til software design
	\item 
\end{itemize}

SE LOGBOG 17-11-2015 \\
Vi har fået rettet BDD, IBD, Grænseflader(2.2 og 2.33), tekst til kondensator (på lager), filter/forstærkerblok, indgangsimpedans og test af forstærkerblok.
\\
\\
Vi har forsøgt at realisere vores hardware fra fumlebræt til hul-board. Dette arbejdes der videre med imorgen (3.12).\\
Vi har rettet UC6 og dertil hørende accepttest. \\
Der er blevet begyndt på software sq diagrammer, og der er blevet programmeret på nulpunktsjusterings funktionen. \\
Der er lavet ændringer i vores GUI pga. rettelser i UC6 efter møde med Peter.
	
	
	
	
% Logbog for den 03-12-2015
\section{Dato: 03-12-2015 }
\hrule

\textbf{Omhandler:} Hardware

\textbf{Ansvarlig:} Lise og Jakob

\textbf{Logbog}
\\
\\
\textbf{Dagsorden:}
\begin{itemize}
	\item Hardware
\end{itemize}

Vi har idag arbejdet med at flytte vores hardware design fra fumlebræt og over til et veroboard, og samtidig gøre det muligt at tilslutte vores tryktransducer til boardet.\\ 
Vi fik både filter og forstærker til at virke på vores veroboard, og vi fik også bekræftet at vores hardwaredel virker med tryktransduceren tilsluttet. Dette testede vi ved at puste i transduceren, og se på skærmen at trykket(spændingen) steg.	\\
Vi skal have kigget på beregninger til implementering af hardware, da vi har påsat en regulator som kontrollerer at der hele tiden kommer 5V som input til transduceren. Den omformer altså de ca. 9V fra batteriet, da tryktransduceren (sandsynligvis) ikke kan holde til et input på 9V.\\
Med et ændret input fra 9V til 5V, vil der også været et ændret max antal mV fra de tidligere 13,5 mV til ca. 7,5 mV. Kig nærmere på alle de beregninger der knytter sig til det.


	
	
% Logbog for den 07-12-2015
\section{Dato: 07-12-2015 }
\hrule

\textbf{Omhandler:} Færdiggørelse af produkt

\textbf{Ansvarlig:} Nina, Lise, Annsofie, Toke, Anders og Jakob

\textbf{Logbog}
\\
\\
\textbf{Dagsorden:}
\begin{itemize}
	\item Færdiggøre hardware
	\item Færdiggøre software
	\item Baggrundsafsnit
\end{itemize}

Vi har fået testet på vores vero-board, og det er nu klar til at blive implementeret sammen med softwaren. \\
Der er blevet skrevet videre på baggrundsafsnittet - herunder om wheatstonebridge.


	
	
% Logbog for den 08-12-2015
\section{Dato: 08-12-2015 }
\hrule

\textbf{Omhandler:} Pre-accepttest

\textbf{Ansvarlig:} Nina, Lise, Annsofie, Toke, Anders og Jakob

\textbf{Logbog}
\\
\\
\textbf{Dagsorden:}
\begin{itemize}
	\item Gennemgang af Use cases og accepttest
	\item Pre-accepttest 
	\item Diverse rettelser
\end{itemize}

Vi har idag gennemgået vores use cases og accepttest i forbindelse med færdiggørelsen af vores software og hardware del. \\
De stemmer nu overens, og vi er klar til at foretage accepttest onsdag.\\
Der er lavet videre på baggrundsafsnittet og skrevet mere til hardware delen(INA114)

	
	
	
% Logbog for den 09-12-2015
\section{Dato: 09-12-2015 }
\hrule

\textbf{Omhandler:} Accepttest

\textbf{Ansvarlig:} Nina, Lise, Annsofie, Toke, Anders og Jakob

\textbf{Logbog}
\\
\\
\textbf{Dagsorden:}
\begin{itemize}
	\item Accepttest
	\item Systembeskrivelse
	\item Projektformulering og afgrænsning
	\item SysML diagrammer (SQ og Klasse diagram)
\end{itemize}

Vi har idag gennemført accepttest for vores færdige produkt, og fået det godkendt af Peter. \\
Vi er nu gået igang med at skrive rapporten.\\
Der er blevet skrevet systembeskrivelse, projektformulering og afgrænsning, samt rettet diverse SysML diagrammer, således at det hele stemmer overens. \\
Vi er klar til, at arbejde videre med rapporten imorgen og de næste mange dage. 


	
	
% Logbog for den 10-12-2015
\section{Dato: 10-12-2015 }
\hrule

\textbf{Omhandler:} Videre arbejde med rapport

\textbf{Ansvarlig:} Nina, Lise, Annsofie, Toke, Anders og Jakob

\textbf{Logbog}
\\
\\
\textbf{Dagsorden:}
\begin{itemize}
	\item Indledning
	\item Resumé
	\item Krav
	\item Metode
	\item Projektgennemførelse
	\item Aktivitetsdiagram
	\item Sekvens diagrammer for Use Cases
	\item Softwareimplementering og test (dokumentation)
\end{itemize}

Vi har idag uddelegeret en række opgaver til gruppens medlemer, som primært drejer sig om rapportskrivning. \\
Der er blevet skrevet en indledning, et resumé, krav, projektgennemførelse, begyndt på metode afsnittet, tilpasset Use Case sekvens diagrammer til tidligere ændringer, begyndt på aktivitetsdiagrammer og så er der blevet skrevet yderligere til softwareimplementering og test.\\
Planen er at blive færdig med rapporten i løbet af mandag, således at vi har hele tirsdag til at læse korrektur osv.\\
Desuden har vi besluttet at tage ud på Skejby mandag d. 14.12.15 kl. 08.00 for at teste vores produkt på en ægte gris. 


	
	
% Logbog for den 11-12-2015
\section{Dato: 11-12-2015 }
\hrule

\textbf{Omhandler:} Uddelegering af afsnit til rapportskrivning

\textbf{Ansvarlig:} Nina, Lise, Annsofie, Toke, Anders og Jakob

\textbf{Logbog}
\\
\\
\textbf{Dagsorden:}
\begin{itemize}
	\item Uddelegering af opgaver
\end{itemize}

Annsofie: Indledning, Arkitektur\\
Nina: Projektgennemførelsen, Arkitektur\\
Lise: Skriver metodeafsnit færdigt\\
Vi mødes 9.30 imorgen (lørdag d. 12.12.15)\\


	
	
% Logbog for den 12-12-2015
\section{Dato: 12-12-2015 }
\hrule

\textbf{Omhandler:} Rapportskrivning

\textbf{Ansvarlig:} Nina, Lise, Toke, Anders og Jakob

\textbf{Logbog}
\\
\\
\textbf{Dagsorden:}
\begin{itemize}
	\item Specifikation og analyse
	\item Arkitektur
	\item Implementering og test (SW) i dokumentationen
	\item GUI i kravspecifikation
\end{itemize}

I specifikation og analyse er der blevet skrevet for hardware, men ikke software.\\
Arkitektur er blevet færdigskrevet. \\
Der blevet arbejdet på implementering og test af software til dokumentationen. Tæt på at være færdig.\\
Der er blevet indsat billeder af GUI ind under ikke-funktionelle krav.\\
Vi snakkede om, at sætte kalibrering via vandsøjle ind under resultater, da det er en nødvendighed før et reelt resultat kan opnåes. Efter beskrivelse af kalibrering vha. vandsølje kan der så komme de reelle resultater.

	
	
% Logbog for den 00-00-2015
\section{Dato: 00-00-2015 }
\hrule

\textbf{Omhandler:} 

\textbf{Ansvarlig:} Nina, Lise, Annsofie, Toke, Anders og Jakob

\textbf{Logbog}
\\
\\
\textbf{Dagsorden:}
\begin{itemize}
	\item 
\end{itemize}

	
	
	
% Logbog for den 00-00-2015
\section{Dato: 00-00-2015 }
\hrule

\textbf{Omhandler:} 

\textbf{Ansvarlig:} Nina, Lise, Annsofie, Toke, Anders og Jakob

\textbf{Logbog}
\\
\\
\textbf{Dagsorden:}
\begin{itemize}
	\item 
\end{itemize}
