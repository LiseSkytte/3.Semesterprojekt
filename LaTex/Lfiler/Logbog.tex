\chapter{Logbog}

\section{Dato: 08-09-2015 }
\hrule

\textbf{Omhandler:} Udarbejdelse af samarbejdskonkrakt 

\textbf{Ansvarlig:} Lise, Nina, Toke, Anders og Jakob

\textbf{Logbog}
\\
\\
Dagsorden:
\begin{itemize}
	\item Udarbejdelse af samarbejdsaftale
	\item Forklaring af Github
	\item Planlægning af vejledermøde
\end{itemize}

Samarbejdsaftalen er udarbejdet og er blevet lagt på Github til godkendelse af alle gruppens medlemmer.\newline 
Lise introducerede Github til gruppens medlemer.\newline 
Annsofie skriver til vores vejleder angående møde mandag d. 14-09-2015 kl. 10.15\newline 
Til næste møde skal "vejledning til 3. semester projekt" være læst og forstået.\newline





\section{Dato: 21-09-2015 }
\hrule

\textbf{Omhandler:} Kravspecifikation

\textbf{Ansvarlig:} Lise, Nina, Annsofie, Toke, Anders og Jakob

\textbf{Logbog}
\\
\\
Dagsorden:
\begin{itemize}
	\item Kravspecifikation

\end{itemize}

Vi vil implementere systole og diastole på vores blodtryksmåler.
En knap i GUI'en til at bestemme om filteret skal være aktivt eller passivt.
En knap i GUI'en til at kalibrere.
En 'Start/record'-knap der også optager lyd de første 30 sek. eks.
EN 'Start'-knap til at starte blodtryksmåling.
Vi dropper login funktionen.
Vi talte om en form for talefunktion, der optager når lægen råber patientens informationer op. Den lydfil skal så kobles til den blodtryksmåling, som foretages fx. i ambulancen. 
Vi skal have defineret: System, borger, UI


\section{Dato: 23-09-2015 }
\hrule

\textbf{Omhandler:} Kravspecfikation 

\textbf{Ansvarlig:} Lise, Nina, Annsofie, Toke, Anders og Jakob

\textbf{Logbog}
\\
\\
Dagsorden:
\begin{itemize}
	\item Definationer
	\item Aktør-kontekst diagram
	\item Use Case diagram
	\item Use Case forløb
\end{itemize}

Ordliste:\\ 
Use Case 1 = UC1 \\
PhysioNet + Analog-Discovery = Borger, som er en Package \\
Analog-Discovery = AD \\


Der er blevet lavet et udkast til AK og UC-diagram. \\
Vi har en Bruger som primær aktør. Vi har lavet en package med PhysioNet og AD, som repræsentere Borger. DAQ er en sekundær aktør, der påvirker Borger. Databasen er også en sekundær aktør. 


Vision for BT-måler:\\
Apperatet skal være fungerende som en "normal" BT-måler med mulighed for visning af kontinuerligt BT med/uden filter. Derudover skal der være mulighed for at optage lydsekvenser direkte på apperatet som kan bruges som metadata til målingen, uden at bruger skal indtaste data på UI. Dette skal implementeres i en Rec. knap der tænder for mikrofonen og optagen en lydsekvens på f.eks. 30 sek, og derefter slukker. Denne lydsekvens kunne bruges til at dokumentere information om Borger. Disse kunne være alder, køn, traume, ABCDE (førsthjælp), mv. på Borger der monitoreres. 
Når blodtryksmålingen gemmes på database eller fil, bliver lydsekvensen gemt sammen med denne. Dette kan anvendes til forskning senere. 


Vi ønsker også at vis dia/sys i UI. 


\section{Dato: 06-10-2015 }
\hrule

\textbf{Omhandler:} Review, tidsplan og hardware  

\textbf{Ansvarlig:} Nina, Lise, Toke, Anders og Jakob

\textbf{Logbog}
\\
\\
Dagsorden:
\begin{itemize}
	\item Review af gruppe 2's KS og AT
	\item Udvikling af tidsplan
	\item Videreudvikling af hardware
\end{itemize}

Der er blevet lavet review af gruppe 2's KS og AT, og givet kommentarer hertil. 
Der er i gruppen blevet udarbejdet en tidsplan, hvori deadlines fra IHA er inkluderet for accepttest, kravspecifikation, design, accepttest med vejleder aflevering. Dertil har gruppen selv tilføjet en række deadlines for software, hardware, rapportskrivning, dokumentation og korrekturlæsning. Alle deadlines er lagt ind i en tabel for at skabe overblik, således at tidsplanen kan benyttes som pejlmærke for hvad vi skal lave og hvornår.
Der er desuden blevet udarbejdet en detaljeret tidsplan, hvor vi beskriver det grundlæggende for udviklingsprocessen.
Anders og Toke arbejdede videre med hardware delen på fumlebrættet.
Forsøger at aftale et møde med Peter mandag eller tirsdag efter kl. 12.00 i uge 43.


\section{Dato: xx-10-2015 }
\hrule

\textbf{Omhandler:}  

\textbf{Ansvarlig:} 

\textbf{Logbog}
\\
\\
Dagsorden:
\begin{itemize}
	\item 
	\item 
	\item 
\end{itemize}