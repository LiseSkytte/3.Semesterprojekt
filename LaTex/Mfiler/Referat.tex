\chapter{Mødereferat}

\section{Dato: 15-09-2015}
\hrule

\textbf{Fremmødte:} Lise, Nina, Annsofie, Toke, Anders og Jakob

\textbf{Fraværende:} 
\\
\\
\textbf{Dagsorden}
\begin{itemize}
	\item Opstart, hvor og hvordan?
	\item Projektets formål (præcicering)
	\item Hjælp til tidsplan (hvor skal vi ligge flest kræfter og hvad vægtes mest)
\end{itemize}

\textbf{Referat} 
\\
Sørg for at fokusere på kravspek først. Hvad skal vores produkt kunne, hvilke funktioner osv.
Derefter kan vi kigge på, hvordan vi skal løse opgaven for systemet.

Use cases: Bedst at lave mange små UC's. 

Der kan være krav til hvor meget HW må fylde.
Frekvenskarakteristik.

KVI: Når man bruger blodtryksmålesystemet så skal man ændre det for hver gang det bruges. En knap med "Nulstil", så der forekommer korrekte målinger. Nulpunktsjustering.

Udgangsspændingen og det tryk der måles.

Kravspek skal forklare systemet som brugeren skal se det.
Der er standarder for hvordan en blodtryksmåler skal se ud. 
Skriv dem ind i kravspec med andet til GUI.
2 scenarier: klinisk eller forskningsmæssigt. 
Klinisk:
Forskningsmæssigt: Vil gerne gemme data. Vi skal have "gemme" aspektet med i opgaven. Det forskningsmæssige aspekt. Gemme blodtrykskurven. Enten som textfile eller i database.
Omformuleres og føjes ting til.
Accepttest laves samtidig med kravspec. 
Accepttest langt hen af vejen bliver simuleret data, som vi selv finder. Evt. physionet.
Vi får også lov at teste i CAVE lab.
Væskesøjle skal bruges til kalibrering, og til at korrigere i softwaren. 
Transduceren skal hentes i lab, hvor vi lavede forsøg i ASB.
Mødetider: onsdage 10-11
torsdag den 24. sep 

Brug tidsplan som et værktøj!!! 
Projektadministration som bilag: Mødereferater, tidsplaner, logbog mv. 



\section{Dato: 24-09-2015}
\hrule

\textbf{Fremmødte:} Lise, Nina, Annsofie, Toke, Anders og Jakob

\textbf{Fraværende:} 
\\
\\
\textbf{Dagsorden}
\begin{itemize}
	\item Opstart, hvor og hvordan?
	\item Projektets formål (præcicering)
	\item Hjælp til tidsplan (hvor skal vi ligge flest kræfter og hvad vægtes mest)
\end{itemize}

\textbf{Referat}
\\
Monitoren kører automatisk - ikke brug for en start-knap!!
Tryksensor og hardware skal ligge imellem DAQ og Borger



\section{Dato: 21-10-2015}
\hrule

\textbf{Fremmødte:} Lise, Nina, Annsofie, Toke, Anders og Jakob

\textbf{Fraværende:} 
\\
\\
\textbf{Dagsorden}
\begin{itemize}
	\item Opstart, hvor og hvordan?
	\item Projektets formål (præcicering)
	\item Hjælp til tidsplan (hvor skal vi ligge flest kræfter og hvad vægtes mest)
\end{itemize}

\textbf{Referat}
\\
Digital filter krav: 
Effekten skal være en udglattende effekt.
Lav et simpelt fir filter.

BDD:
Pressure ved force eller force pr. areal

IBD: 
Smid evt. understående ind under grænseflader
Specifikt på spændingsområder
Specificere outputtet - voltage 0-5 volt eks.
16 bit i stedet for A/D converter
Analog signal sensor out
Analog signal forstærker out

Fast mødetid:
Næste møde med Peter er mandag d. 26. oktober kl. 12.30
Hver onsdag kl. 10.00-11.00
Undtagen onsdag d. 28. oktober og d. 25. november


Lidt tekst omkring hjerte, blod og kredsløb i kroppen.


\section{Dato: 26-10-2015}
\hrule

\textbf{Fremmødte:} Lise, Nina, Annsofie, Toke, Anders og Jakob

\textbf{Fraværende:} 
\\
\\
\textbf{Dagsorden}
\begin{itemize}
	\item Use cases
	\item Kravspec og accepttest
\end{itemize}

\textbf{Referat}
\\
Hardware:
\\
Usikkerhed skal med i dokumentationen. Hvis der er afvigelser osv. Usikkerhed på komponenterne.

Aktør-kontekst diagram:
\\
Specificerer sig kun mod en simulering med Physionet og ikke en rigtig test med tryksensor. 
Som færdigt produkt er det ikke særligt anvendeligt når det kun kan benyttes med Physionet. Vi skal i princippet kun bruge PN til at teste med og så skal vores aktør kontekst diagram og endelige produkt ende ud i et produkt der kan benyttes med tryksensor. 
Vi kan skrive at prototypen er lavet med physionet, men vores AK diagram skal være med tryktranducer og et målbart objekt.


Database:
\\
Hvordan er filformatet for databasen? Det skal uddybes således at bruger kan se hvordan databasen ser ud og fungerer.
Skitse af database.

GUI:
\\
Peter vil gerne have mere fokus på vores GUI i vores kravspec.
Skitser af GUI i kravspec under ikke funktionenelle krav.
Vi kan lave det uden for furps+ - spørg evt. Kim
Han vidste ikke hvordan vores graf vil blive vist. 
Han kunne godt li den, hvor den spiser dens egen hale.
Vi skal specificere hvordan grafen opfører sig i GUI'en - det skal også under ikke-funktionelle krav.


Use cases:
\\
UC2: Ændre evt. "Vis måling med filter"

Kravspec:
\\
Vi skal have skrevet "sidste nulpunktsjusterings tidspunkt" ind i vores ikke-funktionelle krav, ligesom med kalibrering. Det skal gøres tydeligt under kravene, at der er tidspunkt for sidste nulpunktsjustering i monitorvinduet.
MTBF var lavt sat. Vi skal sætte et mere realistisk krav, og så skrive at vi ikke kunne teste på det inden for de rammer vi arbejder inden for.

Accepttest:
\\
Havde han ikke lige fået kigget igennem


\section{Dato: 04-11-2015}
\hrule

\textbf{Fremmødte:} Lise, Nina, Annsofie, Toke, Anders og Jakob

\textbf{Fraværende:} 
\\
\\
\textbf{Dagsorden}
\begin{itemize}
	\item Design
	\item Hardware
\end{itemize}

\textbf{Referat}
\\
Vi skal have sat trykstranduceren som aktør(primær) i AK-diagrammet. Enten erstatte måleobjekt, eller skubbe måleobjekt længere 'ud'.
Dokumentationen er bedømmelses grundlaget i rapporten.
Vi skal have værdi på nulpunktsjustering og kalibrering, således at bruger kan se, hvor meget der er blevet justeret.
Der skal styr på om det er det kalibrerede og nulpunktsjusterede signal der bliver gemt i databasen.
Vi skal have kigget på vores domænemodel og evt. tilføjet alle tre vinduer til modellen, så det stemmer overens med vores kravspecifikation og use cases.
\\
Hardware:
Skal bare deles simpelt op i de to blokke og evt. med batterier(eksitations spænding).
Beskriv hvad der er regnet ud og hvad de forskellige dele skal kunne. 
Fx: "det er har skal være et analogt filter med så og så meget gain."
Vi skal lave en 'opskrift' til hardwaren.
Udregning af komponentværdier skal med i implementering(indeholder også testen).
Test ligger i forlængelse af implementering (V-model).
Datablad skal med som bilag.


\section{Dato: 11-11-2015}
\hrule

\textbf{Fremmødte:} Lise, Nina, Annsofie, Toke, Anders og Jakob

\textbf{Fraværende:} 
\\
\\
\textbf{Dagsorden}
\begin{itemize}
	\item Gennemgang af design
	\item Hardware
\end{itemize}

\textbf{Referat}
\\
Hardware: funktionalitet og grænseflader skal ligge under design. Fx gain kasse - hvor meget gain? Der hvor vi beregner gain, skal ligge under design. Vi skal vise input spændingen og hvordan vi er kommet frem til det. Hvilken båndbredde skal også under design. Der skal beregninger på Gain mv. og så flyttes til design. n
Kondensater udregning skal ikke ligge under design.
Wheatstone bro kan ligges under design eller laves som bilag tilknyttet design. 
Evt. et teori afsnit omkring wheatstone bro, bt osv.
\\
\\
Figurer: Der skal altid være figur tekst. Sammen med figuren og figurteksten, skal man kunne forstå hvad det drejer sig om. Så kan man lave en dokumentations tekst under, hvor man uddyber det mere. 



\section{Dato: 02-12-2015}
\hrule

\textbf{Fremmødte:} Lise, Nina, Annsofie, Toke, Anders og Jakob

\textbf{Fraværende:} 
\\
\\
\textbf{Dagsorden}
\begin{itemize}
	\item Programmering - kalibrering
\end{itemize}

\textbf{Referat}
\\
Nulpunktsjustering: måler relativt til atmosfærisk tryk

Kalibrering: Portene (afmærkninger) på vandsøjlen passer til faste tryk. Tallet som fås ved kalibrering skal deles op ved analyse, så værdierne/spændingen passer. 
Kalibrer i software; softwaren undersøger gennemsnittet af de sidste 10 samples (spændingen). 

Least significant bit: Den skal vi IKKE tænke på.

Digital filtrering: Lavpas filter, omkring 30-50 Hz (Peter siger 30). Der skal undersøges, om der er mulighed for at bruge Matlab (FIR-filter) i C sharp. 

Print: Det er okay at sige, at man var tidspresset. Men undersøg mulighederne for at påsætte flere kondensatorer på Rune's print. Alternativet er at lave et Vevo(?)-board. 

Software-diagrammer: Hvad skal der med? Hvad der har relevans. 

Det som gemmes i databasen skal være rådata. Rådata som i mmHg. 
Selve signalet skal ikke slettes/overskrives. Vi skal tænke mere forskning, hvor man kan gå ind og påvirke blodtrykket og så gemme værdierne. Hver gang der trykkes på "Gem" tilføjes der en ny række med nyt tidsstempel, måleid etc. 
En tidsbegrænsning på "Gem"-funktionen (af blodtryk), så den fx gemmer (record) 10 sekunder frem(brugerdefineret tid, evt. med en øvre grænse på 60 sek.). Mest af alt så man ikke får 2 timers data. "Gem måling" UC skal derfor ændres. Når man gemmer må der gerne kunne tilføjes en kommentar (optional). 
	En default tidsværdi (maks), hvis man fx ved at man skal optage i maks 60 sekunder. Skal ændres inden man trykker på "Gem" i Monitor-vindue. 

