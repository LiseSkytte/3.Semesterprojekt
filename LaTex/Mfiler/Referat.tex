\chapter{Mødereferat}

\section{Dato: 15-09-2015}
\hrule

\textbf{Fremmødte:} Lise, Nina, Annsofie, Toke, Anders og Jakob

\textbf{Fraværende:} 
\\
\\
\textbf{Dagsorden}
\begin{itemize}
	\item Opstart, hvor og hvordan?
	\item Projektets formål (præcicering)
	\item Hjælp til tidsplan (hvor skal vi ligge flest kræfter og hvad vægtes mest)
\end{itemize}

\textbf{Referat} 
Sørg for at fokusere på kravspek først. Hvad skal vores produkt kunne, hvilke funktioner osv.
Derefter kan vi kigge på, hvordan vi skal løse opgaven for systemet.

Use cases: Bedst at lave mange små UC's. 

Der kan være krav til hvor meget HW må fylde.
Frekvenskarakteristik.

KVI: Når man bruger blodtryksmålesystemet så skal man ændre det for hver gang det bruges. En knap med "Nulstil", så der forekommer korrekte målinger. Nulpunktsjustering.

Udgangsspændingen og det tryk der måles.

Kravspek skal forklare systemet som brugeren skal se det.
Der er standarder for hvordan en blodtryksmåler skal se ud. 
Skriv dem ind i kravspec med andet til GUI.
2 scenarier: klinisk eller forskningsmæssigt. 
Klinisk:
Forskningsmæssigt: Vil gerne gemme data. Vi skal have "gemme" aspektet med i opgaven. Det forskningsmæssige aspekt. Gemme blodtrykskurven. Enten som textfile eller i database.
Omformuleres og føjes ting til.
Accepttest laves samtidig med kravspec. 
Accepttest langt hen af vejen bliver simuleret data, som vi selv finder. Evt. physionet.
Vi får også lov at teste i CAVE lab.
Væskesøjle skal bruges til kalibrering, og til at korrigere i softwaren. 
Transduceren skal hentes i lab, hvor vi lavede forsøg i ASB.
Mødetider: onsdage 10-11
torsdag den 24. sep 

Brug tidsplan som et værktøj!!! 
Projektadministration som bilag: Mødereferater, tidsplaner, logbog mv. 
