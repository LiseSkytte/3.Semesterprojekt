\chapter{Resumé}
Projektet tager udgangspunkt i problemstillingen om at kunne monitorere et blodtrykssignal invasivt. Invasive blodtryksmålinger bliver til hverdag benyttet under operationer samt hos svært syge patienter på intensiv afdelinger. 

I dette projekt skal der udvikles et blodtryksmålesystem, der kontinuerligt kan vise en blodtrykskurve på en monitor. Blodtryksmålesystemet skal udover den kontinuerlige visningen kunne kalibreres, nulpunktjusteres, kunne anvende digitalt filter samt gemme data om målingen i en database. 

Systemet består af en softwaredel, der skal løse de funktionelle krav for Blodtryksmålesystemet. Før blodtrykssignalet kommer i kontakt med softwaren, skal signalet behandles af forskellige hardwareelementer. Der er blevet udleveret en A/D konverter og en tryktransducer ved projektets start. Signalet skal ydermere forstærkes og filtreres i en Signalbehandlingsblok, som gruppen selv designer og realiserer på veroboard.

Signalbehandlingsblokken behandler det analoge signal fra tryktransduceren som ønsket. Softwaredelen er blevet udviklet i Visual Studio og opfylder de opstillede krav for Blodtryksmålesystemet. 

Det udviklede Blodtryksmålesystem gennem hele processen er blevet testet. Hardware og software er blevet testet hver for sig via unittest og integrationstest. Systemet er som enhed blevet testet ved udførsel af accepttest. Blodtryksmålesystemet er blevet afprøvet på forskellige måleobjekter, både via simulerede blodtryksmålinger fra PhysioNet, via en In Vitro hjertemodel og til slut på en levende gris.
\\\\


\textbf{Abstract}\\
This paper studies the challenges of invasive measurement of blood pressure. This procedure is used daily in surgery of patients suffering from severe illness, such as those in intensive care units. 

The goal of this assignment is to develop a system, which can measure and display blood pressure continuously on a screen. The system requires functions such as: calibration, zero adjustment, digital filtration and saving measured data on a database. 

The system consists of a software module with the purpose of solving the functional requirements. Before being processed by the software, the blood pressure signal must be amplified and filtered by dedicated hardware elements. These elements will be designed and built by the group.
A handed out pressure transducer and an A/D converter is used to produce and digitize the blood pressure signal. 

It is concluded that the final system of hardware and software meets the requirements specified in the paper.

The system has been tested continuously during the work process. Unit tests have been used to test software and hardware independently. Integration tests and acceptance tests have been performed to test the system in its entirety.
Furthermore the real life application of the system has been tested on a In Vitro heart simulator, simulated APB signal acquired from PhysioNet, and eventually on a live pig. 
