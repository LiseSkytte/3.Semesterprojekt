\chapter{Resumé}
Problemstillingen i dette projekt omfatter en udarbejdelse af program, der har til formål at vise en kontinuerlig blodtrykskurve på en computerskærm. Der er udarbejdet et blodtryksmålesystem, der kan måle blodtrykket invasivt ved at tilslutte et væskefyldt kateter til patientens arterie. Systemet indeholder et elektronisk kredsløb, som forstærker signalet fra en tryktransducer og filtrerer det med et indbygget analogt filter. Derudover indeholder det et program, der har til formal at vise blodtrykskurven som funktion af tiden. Desuden skal løsningen kunne gemme data i en database, samt foretage kalibrering og nulpunktjustering.\\\\
Det udviklede produkt består af software, udviklet i Visual Studio efter 3-lagsmodellen, samt hardware i form af en tryktransducer, forstærker og DAQ. 
Den endelige løsning har resulteret i software, som kan vise en blodtrykskurve kontinuerligt over et interval på 4 sekunder og gemme det i en database. Løsningen er derudover i stand til at filtrere blodtrykket i selve programmet via et digitalt filter, som kan slås til og fra. Ydermere er brugergrænsefladen designet brugervenligt ud fra medicotekniske standarder.\\\\
Gennem en række tests er det konkluderet, at systemet er i stand til at vise en blodtrykskurve kontinuerligt, samt vise systole-, diastole- og pulsværdier med tal. Derudover er det konkluderet, at systemet kan gemme data i en database.

