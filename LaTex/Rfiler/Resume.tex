\chapter{Resumé}
Projektet tager udgangspunkt i problemstillingen om at kunne monitorere et blodtrykssignal invasivt. Invasive blodtryksmålinger bliver til hverdag benyttet under operationer samt hos svært syge patienter på intensiv afdelinger. 

I dette projekt skal der udvikles et blodtryksmålesystem, der kontinuerligt kan vise en blodtrykskurve på en monitor. Blodtryksmålesystemet skal udover den kontinuerlige visningen kunne kalibreres, nulpunktjusteres, kunne anvende digitalt filter samt gemme data om målingen i en database. 

Projektet består af en softwaredel, der skal løse de funktionelle krav for Blodtryksmålesystemet. Før blodtrykssignalet kommer i kontakt med softwaren, skal signalet behandles af forskellige hardwareelementer. Der er blevet udleveret en A/D konverter og en tryktransducer ved projektets start. Signalet skal ydermere forstærkes og filtreres i en Signalbehandlingsblok, som gruppen selv designer og realiserer på veroboard.

Signalbehandlingsblokken behandler det analoge signal fra tryktransduceren som ønsket. Softwaredelen er blevet udviklet i Visual Studio og opfylder de opstillede krav for Blodtryksmålesystemet. 

Det udviklede Blodtryksmålesystemer gennem hele processen blevet testet. Hardware og software er blevet testet hver for sig via unittest og integrationstest. Systemet er som enhed blevet testet ved udførsel af accepttest. Blodtryksmålesystemet er blevet afprøvet på forskellige måleobjekter, både via simulerede blodtryksmålinger fra PhysioNet, via en In Vitro hjertemodel og til slut på en levende gris.


\textbf{Abstract}
