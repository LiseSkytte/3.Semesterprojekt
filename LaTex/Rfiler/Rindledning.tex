\chapter{Indledning}
Måling af det arterielle blodtryk er en af de hyppigst udførte kliniske undersøgelser. Det er et uundværligt led i den kliniske vurdering af patienter. Der findes forskellige metoder til måling af blodtrykket – både invasive og non-invasive metoder, hvor der ved begge tilfælde bestemmes en værdi for det systoliske og diastoliske blodtryk. \\\\
De allerførste målinger af blodtryk blev udført på dyr. Den engelske præst, Stephen Hales (f. 1677), var den første person til at lave invasive målinger på blodtrykket. Han introducerede en ny teknik, der gik ud på at indsætte den ene ende af et messingrør i venstre arterie på en hest, og derefter tilslutte den anden ende af røret til et lodret glasrør således, at han kunne måle blodsøjlens højde og iagttage, hvordan denne steg og faldt i takt med hestens puls. Herved observerede han også, at hestens blodtryk var lavere, når den var i hvile, end når hesten var ophidset. \\\\
Den non-invasive metode vi kender til i dag, hvor der anvendes et kviksølvmanometer, blev introduceret i 1896 af den italienske læge, Scipione Riva-Rocci (f. 1863), som med en okklusionsmanchet gjorde det muligt at måle det systoliske blodtryk. Få år efter i år 1905 blev korotkofflydende for første gang beskrevet af den russiske kirurg, Nikolai S. Korotkov. Det viste sig ligeledes, at få stor indflydelse på nutidens metode til at måle et pålideligt indirekte blodtryk ved den auskultatoriske metode. Dog er den elektroniske blodtryksmåling væsentlig mere udbredt til non-invasive målinger, hvor der anvendes et almindelig blodtryksapparat. \\
Derudover er det i daglig klinisk praksis ofte meget anvendeligt at kunne monitorere en patients blodtryk kontinuert under eksempelvis operationer eller på intensive afdelinger med svært syge patienter. Her bliver blodtrykket målt invasivt, hvor blodtryksmålesystemet er tilsluttet patientens arterier via et væskefyldt kateter. Således kan de hæmodynamiske data monitoreres. Denne metode giver et præcist og kontinuert, grafisk overblik over hjertets arbejde. \\\\
Med forståelsen for hjertet, kredsløbet og blodtryksmålinger er der i dette projekt fokuseret på udarbejdelsen af et monitoreringssystem til at vise en invasiv blodtryksmåling. Der ønskes en udvikling af et fysisk system til at måle trykket fra et måleobjekt, samt et tilhørende program, der grafisk kan udskrive blodtrykssignalet kontinuert som funktion af tiden. Systemet er udviklet til at kunne transformere et fysisk tryk til et analogt signal, som behandles og viderekonverteres til et digitalt signal, der grafisk kan vises i det udviklede program. Projektet er udviklet til forskningssammenhænge, som har været prioriteret i udarbejdelsen af blodtryksmålesystemet samt programmet. Derved er programmet anvendeligt for en forsker, der ønsker at foretage og observere målinger af blodtrykket, samt optage og gemme disse målinger til senere analyse. 
Med udgangspunkt i blodtryksmålersystemet, er der opstillet kravsspecifikation med use cases, og tilhørende accepttest. 
