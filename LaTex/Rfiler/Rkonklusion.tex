\chapter{Konklusion}
Formålet med dette projekt var at udvikle et blodtryksmålesystem, der via et væskefyldt kateter invasivt kunne måle blodtrykket og udskrive dette samt de hæmodynamiske data på en brugergrænseflade. Ud fra problemformuleringen og de opstillede krav er Blodtryksmålesystemet udviklet til at kunne nulpunktjustere, kalibrere samt filtrere blodtrykket via et digitalt filter, der kan aktiveres og deaktiveres. 
Det var et ønske, at Blodtryksmålesystemet skulle være udviklet til forskningsmæssige sammenhænge. Derfor blev dette prioriteret i designfasen, så det færdige system skulle være mere anvendeligt for en forsker, som ønskede at foretage og observere forskellige blodtryksmålinger.\\\\
I den udviklede hardwaredel er det lykkedes at opbygge et elektronisk kredsløb, der behandler et signal, der er transformeret fra et tryk via den udleverede Tryktransducer. Når trykket er blevet transformeret til et analogt signal skal det viderebehandles gennem en Signalbehandlingsblok, som består af en Forstærker og et indbygget analogt Filter. Når signalet har passeret Forstærker og Filter bliver det konverteret i den udleverede DAQ. Her konverteres signalet fra et analogt til et digitalt signal, hvorefter signalet er klar til at blive behandlet i den udviklede software. \\\\
Softwaren er ud fra kravene udviklet til grafisk at kunne udskrive blodtrykssignalet kontinuert som funktion af tiden. Systemet indeholder et C\#-program udviklet i Visual Studio, som var et af de obligatoriske krav. Ud over kalibrering, nulpunktjustering og muligheden for at aktivere og deaktivere det digitale filter, er der udviklet en funktion til at optage det målte blodtrykssignal i en bestemt optagelseslængde, som gør det muligt at gemme de målte data i en database. Dette var ligeledes et krav til softwaren, som er implementeret. Ud over de obligatoriske krav, er det valgt, at systemet ligeledes skulle afbilde værdierne for det systoliske og diastoliske blodtryk med tal samt detektering af puls og udskrivelse af denne. Dette er også succesfuldt implementeret. \\\\
Selve Blodtryksmålesystemets hardware- og softwaredel er løbende testet i udviklingsfasen. Til de endelige test af systemet erfares det, at Blodtryksmålesystemet kan måle og udskrive et blodtrykssignal simuleret fra PhysioNet. Derudover er der også udført test på en In Vitro hjertemodel, hvor systemet ligeledes var i stand til at måle det korrekte tryk, In Vitro maskinen genererede. Dette kunne vi konkludere, da der ligeledes var tilkoblet et blodtryksmonitoreringssystem fra Siemens, som målte de samme data. Den sidste test, der blev foretaget på Blodtryksmålesystemet var In Vivo, hvor vi fik mulighed for at teste på en levende gris på AUH. Ud fra denne test af kan det konkluderes, at vores monitoreringssystem fungerer som ønsket og kan monitorere reelle blodtryksværdier.\\\\
Det endelige Blodtryksmålesystem er et vellykket produkt, der succesfuldt kan udskrive en blodtrykskurve på en grafisk brugergrænseflade. Systemet er udviklet fra problemformuleringen og de opstillede krav, som alle er opfyldt. 
