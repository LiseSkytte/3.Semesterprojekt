\chapter{Projektbeskrivelse}

\section{Projektgennemførelse}
Ud fra den givne projektformulering og tilgang til emnet er der udformet en tidsplan, som indeholder de overordnede deadlines for review og tests givet fra AU.
\begin{figure}[H]
	\centering
	\includegraphics[width=1\textwidth]{Figurer/Snip20151210_74.png}
	\caption{Overordnet tidsplan}
\end{figure}
Udviklingsprocessen, Scrum, er blandt andet blevet brugt i nedbrydning af selve opgaven i delelementer, hvor de vigtigste delopgaver er prioriteret først. Via Scrum inddeles arbejdet i sprints, hvilket tidsplanen ligeledes er blevet. Sprints er forskellige faser, som eksempelvis projektopstart, kravspecifikation, accepttest, design osv.\\\\
I tidsplanen er der markeret et P for det planlagte arbejde, hvor reviews er markeret med bogstavet R. Derudover er der markeret flueben i tidsplanen for udført arbejde. Flueben i parentes repræsenterer, hvornår arbejdet skulle have været færdig, men ikke blev det. Uge 42 er markeret med orange farve, da der var efterårsferie, og her var der heller ikke planlagt arbejde. Uge 47 og 48 er også markeret med orange farve og bogstavet P, da der i denne periode har været planlagt arbejde. Dette arbejde er dog ikke blevet udført grundet eksamenslæsning og eksamen.
\subsection{ASE-modellen}
Projektets udviklingsproces har taget udgangspunkt i ASE-modellen, som ses på nedenstående figur. Denne afspejles desuden også i den overordnede tidsplan. 
\begin{figure}[H]
	\centering
	\includegraphics[width=1\textwidth]{Figurer/asemodel}
	\caption{Udviklingsmodel: ASE-model}
\end{figure}
ASE-modellen er en udviklingsmodel, der tager udgangspunkt i use cases, som defineres i kravspecifikationen i starten af projektarbejdet. ASE-modellen er inspireret af vandfaldsmodellen, hvor projektarbejdet opdeles i faser. Der fastlægges en opgaveformulering, kravspecifikation og systemarkitektur, for derefter at designe, implementere og teste de enkelte moduler i iterationer. Ud fra projektformuleringen specificeres kravspecifikationen som en række use cases, der beskriver de forskellige aktørers interaktion med systemet. Dette giver et overblik over, hvilke krav, der stilles til systemets funktionalitet. Ud fra kravspecifikationen bliver systemets accepttest udarbejdet. Efter kravspecifikationen er fastlagt, udarbejdes systemarkitekturen. Ud fra systemarkitekturen designes systemet ved at nedbryde det efter funktionalitet, som kan bindes til både software og hardware. \\\\
De første step i udviklingen af projektet iflg. ASE-modellen er blevet udarbejdet af alle gruppens medlemmer. Alle har bidraget til projektformuleringen, kravspecifikationen og accepttesten. I begyndelsen var der primært fokus på, at få dette færdiggjort, men der blev allerede her arbejdet på komponentværdier og udkast til hardwaren. I starten af projektarbejdet var det udtænkt, at alle gruppens medlemmer skulle arbejde med alle områder i projektet. Dette kunne ikke udføres rent tidsmæssigt, og derfor var en opdeling af arbejdsopgaver i mindre grupper nødvendig. Det blev delt op i software- og hardwarehold, dog var en del af hardwaren allerede beregnet og implementeret, da denne opdeling fandt sted. Tilbage af hardwaredelen var at lodde det på et veroboard og derefter teste det.\\\\
\subsection{V-modellen}
V-modellen er en udviklingsmodel opdelt i forskellige faser, der beskriver udviklingsfaserne og testfasrrne i projektet sideløbende. Denne model er blevet benyttet sideløbende med ASE-modellen, og fungerer således, at specifikationen af tests foregår sideløbende med udviklingen af selve systemet. Hver fase skal færdiggøres inden næste fase påbegyndes, hvilket også var tiltænkt i projektet. Dette blev dog ikke helt opfyldt i projektet, da der ofte blev rettet i tidligere faser, selvom de reelt skulle have været færdiggjort.
\begin{figure}[H]
	\centering
	\includegraphics[width=1\textwidth]{Figurer/vmodel}
	\caption{V-modellen}
\end{figure}
På ovenstående figur ses V-modellen, hvor den første fase er udvikingen af kravspecifikationen. Hertil udvikles en tilhørende accepttest, som gør det muligt, at tjekke om systemet opfylder de stillede krav. 

\subsection{Deadlines}

\subsection{Mødestruktur}

 




\section{Metode}
Dette afsnit har til formål at beskrive hvilke metoder samt arbejdsredskaber, der er blevet benyttet til udarbejdelsen af software, hardware, rapport og dokumentation for projektet. 
\\\\
Der er blevet anvendt metoder samt viden fra forskellige kurser over 1.- 2.- og 3.semester. I kurset ST3KVI er der blevet benyttet viden om transducerprincippet samt generelt om blodtryksmåling, hvor der er i dette projekt fokuseres på den invasive blodtryksmåling. Fra kurserne ST2ITS2 og ST3ITS3 har viden om databasestruktur samt principperne om tre-lagsmodellen og Observer \& Stragety Pattern været grundlaget for opbygningen af softwaren. Et krav til projektet var, at der skulle kunne aktiveres og deaktiveres et digitalt filter. Til at designe dette filter blev der anvendt viden fra kurset E3DSB. Kurset ST1SUN1 har bidraget med viden om den anatomiske opbygning af hjertet samt blodtryk. Til beregning af komponentværdier, design og implementering af hardwaren er der blevet benyttet metoder og viden fra kurset E2ASB. Principperne om design diagrammer fra kurset I2ISE har bidraget til designafsnittet for hardware og software i dokumentationen. Alle kurserne har tilsammen været nødvendige for at kunne stå tilbage med dette specifkke produkt.        





\section{Specifikation og analyse}

\section{Arkitektur}
\subsection{Design}
\subsection{Implementering og test af SW}
\subsection{Implementering og test af HW}

\section{Resultater og diskussion}

\section{Opnåede erfaringer}

\section{Fremtidigt arbejde}


